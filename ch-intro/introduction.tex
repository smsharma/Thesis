\chapter{Introduction}
\label{ch:intro}

The nature of dark matter (DM) remains one of the major unsolved problems in physics. Originally inferred through its gravitational influence on galaxies and clusters, a rich body of evidence has accumulated over the last four decades firmly establishing its existence. All of the evidence, however, comes from inferring dark matter's presence uniquely through its gravitational effects. Much else about its nature remains an open question: Does dark matter consist  of a fundamental particle? What is its mass? Could there be an entire dark sector, akin to our own Standard Model? How does dark matter interact with the Standard Model?

Answering these questions represents a giant ongoing effort that draws from a rich body of theoretical as well as experimental work, as well as major input from the computational and numerical side. Fortunately, we are at the dawn of a data-driven era in astrophysics and cosmology with great potential to pin down its properties. A plethora ongoing and forthcoming experiments, as well as increasingly open approach to data availability and collaborative analysis, present great promise in pinning down the nature of dark matter. 

Dark matter plays a central role in many subfields of particle physics, astrophysics and cosmology. Understanding the nature of dark matter would have far reaching consequences, as there is no concrete explanation within the frameworks of the Standard Models of particle physics and cosmology. It would represent a paradigm shift and provide major insights into fundamental physics beyond the Standard Model, as well as elucidate the evolution of our Universe and the formation of structures within it. 

While there exists a proliferation of ideas regarding the particle nature of dark matter, the dominant paradigm over the last three decades has been that of the Weakly Interacting Massive Particle (WIMP), which posits dark matter to consist of a particle with weak-scale interactions with the Standard Model, produced through thermal freeze-out in the early universe. This framework dovetails well, for example, with supersymmetric extensions to the Standard Model where candidates for dark matter can emerge naturally. After briefly describing alternative explanations, in this thesis I will focus exclusively on WIMP searches. I will also focus exclusively on astrophysical searches for WIMPs.

This introduction is organized as follows. In Sec.~\ref{sec:evidence} I will summarize the large body of evidence that points to the existence of dark matter. In Sec.~\ref{sec:particledm} I will describe possible explanations underlying the particle nature of dark matter, describing WIMPs and summarizing alternative explanations. Section~\ref{sec:astrodm} will focus on the astrophysical effort to detect and characterize the nature of DM, in particular using gamma-ray data, and I will describe the theoretical and experimental tools available to us and the status of the field in this context. Finally, in Sec.~\ref{sec:summary} I will describe the organization of the rest of this thesis. Wherever possible, I will touch upon the historical developments that have paved the way for our current state of understanding of dark matter and our efforts to look for it.

\section{Evidence for Dark Matter}
\label{sec:evidence}

In this section, I describe the evidence we have for the existence of dark matter in our Universe. Although dark matter had its inception and development in the 20th century, several scientific problems that philosophically encapsulate the dark matter problem as we are confronted with today had their roots in earlier times. 
For example, the Aristotelian geometric view of an immutable Universe with the Earth as its center on the other hand offered a clean framework that did not call for additional invisible matter or celestial objects, and was the orthodox viewpoint until Renaissance astronomers conclusively refuted it with observations. Galileo, in the face of significant resistance from the Catholic Church, arguably played the largest role in this and discovered much that was previous unknowable -- including understanding the make-up of the Milky Way as consisting of individual stars rather than nebulous clouds, and observing Saturn's rings and Jupiter's four largest moons. These observations are very much in the spirit of modern dark matter searches -- demonstrating that the Universe can contain invisible forms of matter, and that scientific inquiry and technological developments can play a big role in revealing it to us.

% \subsection{Dynamical Evidence}

Evidence for some yet-unknown form of matter started piling up in the early 19th century. Lord Kelvin introduced the idea of applying the ``theory of gases'' to the Milky Way, describing stars as gas particles acting under the influence of gravity, in the process wondering whether a large number of stars may actually be dark bodies. In 1922, Dutch astronomer Jacobus Kapteyn described for the first time a predictive model for the distribution of matter in the Milky Way, describing the stars as particles in a virialized system. Kapteyn used this method to obtain the local matter density in term of the the observed stellar mass, diving out the gravitational mass by the number of stars observed, extrapolating the stellar luminosity function down below that observed. Kapteyn's student Jan Oort as well as several others during this time, including Jeans, Lindblad and Opik were able to derive estimates for the density of matter in the local neighborhood, usually claiming that an excess above the observed stellar mass could be accounted for by the extrapolation of the stellar luminosity function down to very faint stars.

In 1933, Swiss-American astronomer Fritz Zwicky studied redshift data on galaxy clusters collected by Hubble and Humason, noticing large velocity dispersions in eight galaxies within the Coma cluster and applying the virial theorem to estimate its mass. Zwicky predicted the dispersion by using the number of observed galaxies, average mass of a galaxy and its extent, finding a value closer to 80 km/s. This was in stark conflict with the observed line-of-sight velocity dispersion of 1000 km/s. From this, Zwicky concluded that ``If this would be confirmed, we would get the surprising result that dark matter is present in much greater amount than luminous matter''. Zwicky's use of the term \emph{dark matter} was in continuity with usage by astronomers around that time. In subsequent work, Zwicky was able to refine his estimates, confirming a very high mass-to-light ratio in the Coma cluster. An analysis of the Virgo cluster by Sinclair Smith in 1936 again pointed to a very high mass-to-light ratio in that system. In both cases, the astronomers put forward potential explanations in terms of ``clouds of low-luminosity internebular material''

Although this represented a conundrum, there was widespread consensus that more information would be needed to understand what was going on. Historically, velocity rotation curves -- showing the circular velocity profiles of stars in a galaxy while varying the distance from the galactic center -- did the most to convince the scientific community to the existence of large amounts of non-luminous matter in galaxies. The idea here is as follows. Standard Newtonian theory dictates that the circular velocity of stars is given by $v_c(r) = \sqrt{GM/r}$, where $r$ is the radial distance, $M$ the enclosed mass and $G$ the universal gravitational constant. In the region beyond the galactic disk (dictating the observed extent of the galaxy), we expect the enclosed mass to be constant, and the Gauss' law dictates that the circular velocity fall as $v_c \propto r^{-1/2}$. The discovery of the 21-cm emission line in 19something heralded a new era in radio astronomy, and enabled the accurate measurement of rotation curves, starting with the nearby galaxies M31 and M33 and indeed our own Milky Way. Building upon the work of several others, Kent Ford and Vera Rubin, and [person] obtained rotation curves for nearby galaxies, pointing to the approximate flattening out of rotation curves at larger radii contrary to above expectation. This implied that the mass continues to increase as $M \propto r$, pointing to the existence of additional unobserved `dark' mass beyond the visible component. From this, the dark matter density can be inferred to roughly scale as $\rho(r) \propto 1/r^2$. A large number of observations on galactic and cluster scales since then have strengthened the case for the existence of dark matter. In the left panel of Fig.~\ref{key}, I show the measured rotation curves for the Milky Way, and expectation from a disk-like component inferred from baryonic matter, as well as an additional dark matter component. It can clearly be seen that the additional component is required to match the observed rotation curves [describe plot].

% Bullet cluster?

% \subsection{Cosmological Evidence}

Cosmology provides substantial evidence for the presence of dark matter in our Universe. $\Lambda$CDM, often referred to as the standard model of cosmology, is associated with the presence of dark energy ($\Lambda$) and cold dark matter (CDM), is able to account for a plethora of cosmological observations, including the structure and existence of the cosmic microwave background (CMB) radiation, large-scale structure distribution of matter, accelerating expansion of the Universe as measure from ... and relic elemental abundances. 

The CMB by itself provides irrefutable evidence for dark matter. The main relevant observable is the angular scale of inhomogeneities in the temperature distribution of the CMB -- the TT angular power spectrum. The spectrum largely consists of a set of peaks, each peak giving us an angular scale with a particularly significant contributions to the temperature fluctuations. The leading physical effect behind these are acoustic oscillations in the baryon-photon fluid during photon decoupling. Early on, photons and baryons were electromagnetically coupled, and non-baryonic dark matter can generate gravitational potential wells that can pull in the baryon-photon fluid, which go along for the ride. The photon pressure acting against these wells give rise to a tower of modes which can provide information about the relative composition of baryonic and non-baryonic (dark) matter. Detailed physics is nuanced, see Wayne Hu's tutorials for introduction. In short, the position of the first peak, for example, tells us about the curvature of the universe (and hence how much total stuff there is in it), while the ratio of heights between the first and second peaks tells us how much of the matter is baryonic (ordinary matter). The third peak can shed insights into non-baryonic dark matter. The WMAP satellite, while not able to fully resolve the third peak, was already able to say that dark matter makes up the majority of the matter budget in the Universe, finding.... Since then, Planck has been able to pin down.... The right panel of Fig.~\ref{key} shows the Planck TT spectrum, along with the theoretical predictions.

Observations of the distribution of galaxies over a large range of scales provide further strong evidence for the existence of non-baryonic dark matter... weak lensing... bullet cluster.

\section{(Particle) Nature of Dark Matter}
\label{sec:particledm}

Although the existence of dark matter is incontrovertible, its nature remains largely a mystery. These days, it is often implicitly assumed that when people are talking about detecting dark matter, say at a particle collider or in gamma ray data, that this refers to a dark matter \emph{particle}. As touched upon above, this was by no means always the case -- early usage and references to dark matter usually implied the existence of generic dark objects that would be too faint to be observed, perhaps dim stars or internebular material. The transition in usage from an adjective to a noun was a result of sociological changes within the particle physics and astrophysics communities, bringing the two closer after the missing mass problem had been firmly accepted in the 1970s. All the evidence amassed since then is consistent with dark matter being made up of a fundamental particle, or perhaps even the existence of an entire dark sector consisting of many particles with a rich set of properties and interactions.

Within the Standard Model, neutrinos, by virtue of being stable, electrically neutral particles and not interacting strongly, contain some of the essential ingredients for a particle dark matter candidate, and were discussed in this context early on. Cosmological effects of neutrinos were discussed throughout the 1960s and 1970s, pioneered by the work of Zeldovich and others, and their implications for the missing mass observed on (super-)galactic scales was discussed by Weinberg and others in the the late 1970s. Early simulations during the 1980s eventually showed that hot (relativistic) and cold (non-relativistic) particle dark matter would lead to very different outcomes during structure formation, leading to collapse and formation of larger structures ("top-down") in the former case and a hierarchical "bottom-up" approach in the latter. Neutrinos, being very light thermal relics, would be extremely relativistic during structure formation, and early surveys of the local Universe were able to quickly disfavor their role as dark matter candidates. Nevertheless, along the way neutrinos served as a gateway particle in understanding the implications of particle physics to observations on galactic, cluster and cosmological scales.

With no reason to be confined to the Standard Model, supersymmetry posits that nature may contain a spacetime symmetry relation bosons and fermions, requiring that for ever boson a fermion with the same quantum numbers must exist (and vice versa). This leads to the prediction of several new electrically neutral particles uncharged under the strong force. If some of these were stable, they could have played an important role in the history of our Universe and could conceivably make up some portion of dark matter. Supersymmetry took its modern form with a paper by Savas Dimopolous and Howard Georgi, who introduced the MSSM. In the MSSM, superpartners of the $Z$ boson, photon and two Higgses mix to form four particles, today known as neutralinos, which have since become arguably the most-discussed dark matter candidate. 

% There would have to be a mechanism preventing the lightest neutralino from decaying quickly after being created, 

% (some discussion of R-parity), first introduced by Pagels and Primack.

(how SUSY  is strongly motivated in its own right)

(Discussion of axions)

What do we know about dark matter? Gelmini.

\subsection{The WIMP Paradigm and The Search For}

Searches for WIMPs are generally organized into a few categories depending on the interaction probed and experimental direction method. Collider searches focus on... and look for... . Terrestrial experiments, known as direct detection, look for... . Lastly, astrophysical search, indirect detection ... look for... .  In this thesis, I will focus on this last category of astrophysical dark matter searches, or ``indirect detection''.The search for dark matter using astrophysics is closely tied to understanding the nature of astrophysics itself -- ``to understand your signal you must understand your background'', and I'll spend a substantial portion of this thesis on this task as well. 


(Commonality in particle DM candidates)

(Freeze out calculation)

(Gunn tremaine and Lee weinberg)

Axions, SNu, PBH, LDM, dark photons.

\section{Astrophysical Detection of Dark Matter}
\label{sec:astrodm}

\subsection{Tools for Indirect Detection}

\subsection{Dark Matter and Gamma Rays}

\section{Summary}
\label{sec:summary}


% \begin{itemize}
% \item History of dark matter

% \item Particle dark matter
% \item The case for WIMPs 
% \item Dark matter searches
% \end{itemize}