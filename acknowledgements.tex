% \lettrine[lines=3]{G}{oing} through graduate school has been an immensely rewarding and humbling experience, and it is my pleasure to thank . The work in this thesis has only been possible due to an amazing group of collaborators

%  I've been fortunate to work with an amazing group of people while at Princeton, and  

% As a fundamentally human endeavor, 

\lettrine[lines=3]{F}{irst} and foremost, I would like to thank my advisor Mariangela Lisanti. I walked into Mariangela's office for the first time four years ago as a student in the high energy experiment group interested in learning more about collider phenomenology. I was so impressed by Mariangela's enthusiasm and dedication, as well as the breadth of interesting physics she was engaged in, that I decided very soon after that I wanted to work with her. This is a decision I am very glad to have made, and I'm grateful to Mariangela for having taken a chance on me. Mariangela is an amazing scientist and mentor. Her prowess as a physicist is extremely impressive, and her unique and refreshing approach to research is something I continually strive to emulate. As a mentor, Mariangela goes far above and beyond for her students and has always been extremely generous to me with her time. For her constant encouragement and support (memorably during my $\sim$biannual angst sessions); for instilling in me a strong sense of ethics in doing science; and for pushing me when I needed it---my words here will not do justice to how indebted I am to her. Princeton is extremely lucky to have her.

I owe the existence of this thesis to my good fortune in getting to work with many amazing collaborators while at Princeton. The first of these is Ben Safdi. I first got to know Ben while he was a (much older) fellow graduate student at Princeton. He strongly encouraged me to talk to Mariangela and to consider working with her, and I had the good sense to take this advice to heart. This turned out to be only the first of his many contributions to my graduate school experience---every one of the projects included in this thesis was done with him. His insights were absolutely key to the success of these projects, and many of the techniques pioneered by him have become standard elements of my research toolbox. My contemporary Nick Rodd also deserves a special shout out. We have worked on several projects together---three of which are included in this thesis---and it has always been a pleasure to work and chat science with him. I also want to thank Lina Necib, who played an essential role in the high latitude \emph{Fermi} point source analysis presented in this thesis. Lina has been an enduring friend and source of good advice in navigating life and academia. I am no less indebted to my other collaborators, work with whom I sadly cannot include in this thesis: David Alonso, Laura Chang, Tim Cohen, Jo Dunkley, Yoni Kahn, Gordan Krnjaic, Samuel Lee, Tim Lou and Tim Tait. I have learned a lot from each one of you, and I look forward to our future collaborations. 

%Research aside, navigating academia can be confusing and daunting. 
I'm indebted to Tim Cohen, Jo Dunkley and Tim Tait for helping me with postdoc applications and for being wellsprings of advice and encouragement throughout the process. I'm also grateful to Jo for making me feel welcome in the wider cosmology community at Princeton and for agreeing to be on my thesis committee. Peter Meyers deserves a huge thanks for carefully reading the entirety of this thesis and for being on my pre-thesis committee. 

I started grad school intending to focus on experimental high energy physics, and owe many thanks to the Princeton hep-ex group: to Jim Olsen, for supervising the initial stages of my research and my experimental project; to Dan Marlow, without whom I'd probably be in Pasadena right now; and to Chris Tully, for being on my thesis and pre-thesis committees. 

Over my time at Princeton, I have seen the pheno group evolve from just a handful of people into a proper \emph{group}. Hanging out and chatting fizziks with the group has always been a pleasure. A big thank you to the postdocs: Yoni Kahn for giving me a taste of model building and Oren Slone for introducing beer to the Pheno \& Vino seminars. To Mariangela's other grad students Laura Chang and Matt Moschella---kinda jealous you guys get to stay. Laura is an extremely sharp physicist and I have learned a great deal from our collaboration already; I look forward to doing more cool physics together. I'm also grateful to Laura for carefully combing through parts of this thesis and substantially improving the quality of writing in it (and for continually encouraging me to write better!). Matt---I look forward to following your progress at Princeton and beyond.

Thanks go to the denizens of Jadwin Hall without whom life as a graduate student would have been much more dull. To Farzan Beroz, meeting whom at the Princeton Physics Open House gave me hope that it was possible to Live One's Best Life as a grad student; To Ilya Belopolski, whose blasting Ke\$ha full-volume at my first Friday Beer instantly made me feel at home in Jadwin; to Mallika Randeria and Tom Hazard, for the days and nights spent studying for prelims back in first year, and for their enduring friendship since; and to Justin Ripley, for the many mundane and deep, but always fascinating, conversations. %Here's to us one day crossing over from stage-4.5 nihilism to stage-5 fluidity.

There is no doubt that Princeton Physics has the best staff out there, and I sincerely appreciate all their hard work which keeps the wheels turning in the department. These  past five years would have gone very differently, and certainly for the worse, were it not for the help and support of a fantastic team of administrative staff: Toni Sarchi, Kate Brosowsky, Kate Hare, Jessica Heslin and Barbara Mooring. More importantly, seeing them around Jadwin was always guaranteed to brighten my day. 
Although as a theorist my interaction with the A-floor staff was unfortunately limited, seeing Ted, Darryl and Julio always cheered me up. I'm grateful to the Computational Sciences and Engineering Support staff for maintaining and providing timely assistance with the usage of the Princeton computing clusters, enabling the computational needs of the work presented in this thesis.

A very special thanks to Jonathan Balkind for his friendship, support and the bottomless well of inside jokes shared over the last five years. To Jaan Altosaar, Jos\'e Ferreira, Anna FitzMaurice, Raghav Sethi and Maciej Halber---there really isn't just one thing I can mention here and thank you for. I sincerely hope we continue to stay close after grad school.

I am grateful to my Petrean mates for their enduring long-distance friendship. It was a pleasure having Aidan Chan, Duncan Goudie, Oli Kim, Sebastian Koch and Qian Chen visit at various points during my PhD. A special shout out to Gayathri Kumar for always rounding up the peeps and making me feel at home during my impromptu visits to the UK. My visits to Oxford to see Gabija \u{Z}emaityt\.{e} were always an absolute delight, and I deeply cherish our get-togethers. Thanks are also in order to Jack Collins for all the content.

To my parents and my brother---\emph{spasibo} for your unconditional love and support over the last two+ decades. Mama and bapa---there is no chance I would be where I am today without the opportunities you've afforded me through your hard work and sacrifice.

Finally, although I've already mentioned Laura in several contexts, I'm also ever grateful to her for her love, support and simply for being. Laura, I never remotely expected to meet someone like you during grad school, and feel so lucky that I get to call you my partner and my best friend. I have learned so much from you, grown so much with you and am excited for what the future has in store for us. Despite the abundance of cold dark matter in the Universe (see Sec.~\ref{sec:evidence}) and in this thesis, you make the Universe, and my life within it, seem infinitely less cold and dark.

\sectionline

